\documentclass[12pt, a4paper]{article}
\author{Aaron Berger, Christopher Chute, Matthew Stone}
\title{Notes from Mastermind Research}
\usepackage[bottom=1.5in, left=1in, right=1in, top=1.1in]{geometry}
\usepackage{amsmath, amsthm, amssymb}
\newtheorem{lem}{Lemma}
\usepackage{graphicx}
\usepackage{setspace}
\usepackage{enumitem}
\usepackage{booktabs}
\usepackage{footnote}

% \begin{header footer formatting}
\usepackage{fancyhdr}
\setlength{\headheight}{48pt}
\pagestyle{fancyplain}
\lhead{Mastermind Project: \textit{Notes}\\\today{}}
\rhead{AB, CC, MS}
\rfoot{\thepage}
\cfoot{}
% \end{header footer formatting}

% remove ``*'' for numbered theorems
\newtheorem{theorem}{Theorem}
\newtheorem{lemma}{Lemma}

% \begin{pretty i-hat and j-hat}
\newcommand*{\ihat}{\hat{\imath}}
\newcommand*{\jhat}{\hat{\jmath}}
\newcommand*{\jwidehat}{\widehat{\jmath}}
% \end{pretty i-hat and j-hat}

% \begin{pretty code or pseudocode}
\usepackage{listings}
\usepackage{courier}
\lstset{
         basicstyle=\footnotesize\ttfamily,
%        numbers=left,               
         numberstyle=\tiny,          
%        stepnumber=2,              
         numbersep=5pt,              
         tabsize=4,                 
         extendedchars=true,         
         breaklines=true,            
         keywordstyle=\color{red},
         frame=tblr,% Set the borders, tblr == (top, bottom, left, right)
%        keywordstyle=[1]\textbf,    
%        keywordstyle=[2]\textbf,   
%        keywordstyle=[3]\textbf,   
%        keywordstyle=[4]\textbf,  
         stringstyle=\color{white}\ttfamily,
         showspaces=false,          
         showtabs=false,           
         xleftmargin=17pt,
         framexleftmargin=17pt,
         framexrightmargin=5pt,
         framexbottommargin=4pt,
%        backgroundcolor=\color{red},
         showstringspaces=false            
 }
\lstloadlanguages{C}
% Example Usage:
% \lstinputlisting{pseudocode.txt}
% \end{pretty code or pseudocode}

% ********************************* END OF PREAMBLE ***********************************

\begin{document}
This proof will follow a similar path as [Grebinski and Kucherov (optimally...additive model)].
We want 
\begin{equation*}
	\sum_{i=0}^{\frac{x}{2}}\text{Pr}[\text{Eq}(q,v_1) = i \wedge \text{Eq}(q,v_2) = i] \leq \sum_{i=0}^x\text{Pr}[\text{Eq}(q,v_1) = i]\cdot\text{Pr}[ \text{Eq}(q,v_2) = i]
\end{equation*}
We will show this in an appendix or something.
\begin{align*}
&\sum_{i=0}^x\text{Pr}[\text{Eq}(q,v_1) = i]\cdot\text{Pr}[ \text{Eq}(q,v_2) = i] \\
&\le \sum_{i=0}^x\text{P}_\text{max}(\text{Eq}(q,v_1))\cdot\text{Pr}[ \text{Eq}(q,v_2) = i] \\
&= \text{P}_\text{max}(\text{Eq}(q,v_1)) \cdot \sum_{i=0}^x\text{Pr}[ \text{Eq}(q,v_2) = i] \\
&= \text{P}_\text{max}(\text{Eq}(q,v_1))
\end{align*}
P$_\text{max}$ will always be bucket 0 unless $x=n$, in which case it's bucket 1. This is proved by taking a ratio of consecutive terms to show the sequence is decreasing for all other values of \#of hits. [insert actual equation here]
\begin{align*}
\text{Pr}[\text{Eq}(q,v_1) =0] = \left(1-\frac{1}{n}\right)^x \\
\text{Pr}[\text{Eq}(q,v_1) =1 | x= n] = \left(1-\frac{1}{n}\right)^{n-1} \\
\text{So we can write P$_\text{max} \leq \left(1-\frac{1}{n}\right)^{\text{min}(x,n-1)}$}
\end{align*}
So the sum of the probability that two vectors have the same response on all $s$ questions is:
\begin{equation*}
\sum_{x=2}^n(\text{\# of reduced pairs that disagree in $x$ spots})(\text{Probability this pair is in the same bucket}) 
\end{equation*}
\begin{align*}
&=\sum_{x=1}^n \binom{n}{x}n^x(n-1)^x \left(1-\frac{1}{n}\right)^{s\cdot{\text{min}(x,n-1)}} \\
&\le \sum_{x=1}^n n^{3x} \left(1-\frac{1}{n}\right)^{s\cdot{\text{min}(x,n-1)}} \\
&= \sum_{x=1}^n n^{3x} \left(1-\frac{1}{n}\right)^{(4n\log n){\text{min}(x,n-1)}} \\
&< \sum_{x=1}^n n^{3x} \left(\frac{1}{e}\right)^{(4\log n){\text{min}(x,n-1)}} \\
&= \sum_{x=1}^n n^{3x} \left(\frac{1}{n}\right)^{4\cdot{\text{min}(x,n-1)}} \\
&\le \sum_{x=1}^n \frac{1}{n} \\
&\le 1
\end{align*}
\end{document}






























