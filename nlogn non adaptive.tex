\documentclass[12pt, a4paper]{article}
\author{Aaron Berger, Christopher Chute, Matthew Stone}
\title{Notes from Mastermind Research}
\usepackage[bottom=1.5in, left=1in, right=1in, top=1.1in]{geometry}
\usepackage{amsmath, amsthm, amssymb}
\newtheorem{lem}{Lemma}
\usepackage{graphicx}
\usepackage{setspace}
\usepackage{enumitem}
\usepackage{booktabs}
\usepackage{footnote}

% \begin{header footer formatting}
\usepackage{fancyhdr}
\setlength{\headheight}{48pt}
\pagestyle{fancyplain}
\lhead{Mastermind Project: \textit{Notes}\\\today{}}
\rhead{AB, CC, MS}
\rfoot{\thepage}
\cfoot{}
% \end{header footer formatting}

% remove ``*'' for numbered theorems
\newtheorem{theorem}{Theorem}
\newtheorem{lemma}{Lemma}

% \begin{pretty i-hat and j-hat}
\newcommand*{\ihat}{\hat{\imath}}
\newcommand*{\jhat}{\hat{\jmath}}
\newcommand*{\jwidehat}{\widehat{\jmath}}
% \end{pretty i-hat and j-hat}

% \begin{pretty code or pseudocode}
\usepackage{listings}
\usepackage{courier}
\lstset{
         basicstyle=\footnotesize\ttfamily,
%        numbers=left,               
         numberstyle=\tiny,          
%        stepnumber=2,              
         numbersep=5pt,              
         tabsize=4,                 
         extendedchars=true,         
         breaklines=true,            
         keywordstyle=\color{red},
         frame=tblr,% Set the borders, tblr == (top, bottom, left, right)
%        keywordstyle=[1]\textbf,    
%        keywordstyle=[2]\textbf,   
%        keywordstyle=[3]\textbf,   
%        keywordstyle=[4]\textbf,  
         stringstyle=\color{white}\ttfamily,
         showspaces=false,          
         showtabs=false,           
         xleftmargin=17pt,
         framexleftmargin=17pt,
         framexrightmargin=5pt,
         framexbottommargin=4pt,
%        backgroundcolor=\color{red},
         showstringspaces=false            
 }
\lstloadlanguages{C}
% Example Usage:
% \lstinputlisting{pseudocode.txt}
% \end{pretty code or pseudocode}

% ********************************* END OF PREAMBLE ***********************************

\begin{document}
This proof was inspired by [Grebinski and Kucherov (optimally...additive model)]. This holds for $k\geq n$. \\
Claim:
\begin{equation*}
	\sum_{i=0}^{\frac{x}{2}}\text{Pr}[\text{Eq}(q,v_1) = i \wedge \text{Eq}(q,v_2) = i] \leq \left(1-\frac{1}{k}\right)^x
\end{equation*}
Case 1: $x \leq k-3$. We know 
\begin{align*}
\text{Pr}[\text{Eq}(q,v_1) = i \wedge \text{Eq}(q,v_2) = i] &= 
\text{Pr}[\text{Eq}(q,v_2) = i] \cdot
\text{Pr}[\text{Eq}(q,v_1) = i | \text{Eq}(q,v_2) = i]   \\
& = \binom{x}{i}\left(\frac{1}{k}\right)^i \left(\frac{k-1}{k}\right)^{x-i} \cdot \binom{x-i}{i}\left(\frac{1}{k-1}\right)^{i}\left(\frac{k-2}{k-1}\right)^{x-2i} \\
&=\frac{x!}{(k-2)^{2i}(i!)^2(x-2i)!}\left(\frac{k-2}{k}\right)^x \\
&\leq \frac{x^{2i}}{(k-2)^{2i}(i!)}\left(\frac{k-2}{k}\right)^x.
\end{align*}
Now we have 
\begin{align*}
\sum_{i=0}^{\frac{x}{2}}\text{Pr}[\text{Eq}(q,v_1) = i \wedge \text{Eq}(q,v_2) = i] \left(\frac{k}{k-1}\right)^x 
&\leq \sum_{i=0}^{\frac{x}{2}} \frac{x^{2i}}{(k-2)^{2i}(i!)}\left(\frac{k-2}{k}\right)^x \left(\frac{k}{k-1}\right)^x \\
&=\sum_{i=0}^{\frac{x}{2}} \frac{x^{2i}}{(k-2)^{2i}(i!)}\left(1-\frac{1}{k-1}\right)^x \\
&<\left(1-\frac{1}{k-1}\right)^x ~ \sum_{i=0}^{\infty} \left(\frac{x^2}{(k-2)^2}\right)^i\frac{1}{i!} \\
&<e^{-\frac{x}{k-1}}\cdot e^\frac{x^2}{(k-2)^2} \\
&<1\quad \text{ for } x \leq (k-3).
\end{align*}
Multiplying both sides of this inequality by $(1-1/k)^x$ results in the claim. 
\clearpage
Case 2: $k-2 \leq x \leq k$
\begin{equation*}
\sum_{i=0}^{\lfloor x/2 \rfloor} \binom{x}{i}\binom{x-i}{i}(k-2)^{-2i}=\sum_{i=0}^{\lfloor x/2 \rfloor} \frac{\frac{x(x-1)\cdots(x-2i+1)}{(k-2)^{2i}}}{i!^2}
\end{equation*}
For $x=k-2, k-1, \text{or } k$:\\
\\
For $i\ge3$, $x(x-1)\cdots(x-2i+1)\le k(k-1)\cdots(k-2i+1)\le(k-2)^{2i}$. Thus, 
\begin{align*}
\sum_{i=3}^{\lfloor x/2 \rfloor} \binom{x}{i}\binom{x-i}{i}(k-2)^{-2i} & \le \sum_{i=3}^{\lfloor x/2 \rfloor} \frac{1}{(i!)^2} \\
& \le \sum_{i=3}^\infty \frac{1}{(i!)^2} \\
& < .0296\\
\end{align*}
\begin{align*}
\sum_{i=0}^{2} \binom{x}{i}\binom{x-i}{i}(k-2)^{-2i} & = 1+\frac{x(x-1)}{(k-2)^2}+\frac{x(x-1)(x-2)(x-3)}{4(k-2)^4} \\
& \le 1+\frac{k(k-1)}{(k-2)^2}+\frac{k(k-1)(k-2)(k-3)}{4(k-2)^4} \\
& < 2.5536 \text{ for } k\ge 14
\end{align*}
Thus putting the two together gives
\begin{align*}
\sum_{i=0}^{\lfloor x/2 \rfloor} \binom{x}{i}\binom{x-i}{i}(k-2)^{-2i} & \le 2.5832 \\
& < \left(1+\frac{1}{k-2}\right)^{k-2} \text{ for } k\ge14\\
& \le \left(1+\frac{1}{k-2}\right)^{x}
\end{align*}

\clearpage
So the sum of the probability that two vectors have the same response on all $s$ questions is:
\begin{equation*}
\sum_{x=1}^n(\text{\# of reduced pairs that disagree in $x$ spots})(\text{Probability this pair is in the same bucket}) 
\end{equation*}
\begin{align*}
&=\sum_{x=1}^n \binom{n}{x}n^x(n-1)^x \left(1-\frac{1}{k}\right)^{s\cdot x} \\
&\le \sum_{x=1}^n n^{3x} \left(1-\frac{1}{k}\right)^{s\cdot x} \\
&= \sum_{x=1}^n n^{3x} \left(1-\frac{1}{k}\right)^{(4k\log n)x} \\
&< \sum_{x=1}^n n^{3x} \left(\frac{1}{e}\right)^{(4\log n)x} \\
&= \sum_{x=1}^n n^{3x} \left(\frac{1}{n}\right)^{4\cdot x} \\
&\le \sum_{x=1}^n \frac{1}{n} \\
&\le 1
\end{align*}
\end{document}






























