\documentclass[12pt, a4paper]{article}
\author{Aaron Berger\qquad Christopher Chute\qquad Matthew Stone}
\title{Query Complexity of Mastermind Variants}
\usepackage[bottom=1.5in, left=1in, right=1in, top=1.5in]{geometry}
\usepackage{amsmath, amsthm, amssymb}
\newtheorem{lem}{Lemma}
\usepackage{graphicx}
\usepackage{setspace}
\usepackage{enumitem}
\usepackage{booktabs}
\usepackage{footnote}

% 	 \begin{header footer formatting}
%    \usepackage{fancyhdr}
%    \setlength{\headheight}{48pt}
%    \pagestyle{fancyplain}
%    \lhead{Mastermind Project: \textit{Notes}\\\today{}}
%    \rhead{AB, CC, MS}
%    \rfoot{\thepage}
%    \cfoot{}
% 	 \end{header footer formatting}

% remove ``*'' for numbered theorems
\newtheorem{theorem}{Theorem}
\newtheorem{lemma}{Lemma}

% \begin{pretty i-hat and j-hat}
\newcommand*{\ihat}{\hat{\imath}}
\newcommand*{\jhat}{\hat{\jmath}}
\newcommand*{\jwidehat}{\widehat{\jmath}}
% \end{pretty i-hat and j-hat}

% ********************************* END OF PREAMBLE ***********************************

\begin{document}
\maketitle

\begin{abstract}
We analyze variants of the popular board game Mastermind. In this two-player game,
the codebreaker submits queries with the goal of identifying a hidden
sequence, constructed at the beginning of the game by the codemaker. At each step,
the codebreaker receives feedback in the form of ``black'' and ``white'' hits
and incorporates the response into his next guess. We discuss asymptotics for the
number of guesses needed to identify an unknown $n$-vector constructed from $k$
possible colors. We look at strategies that receive two-color responses, as well as
black hit-only responses. We consider both allowing and prohibiting repeated
colors in the hidden sequence, and we analyze both adaptive and non-adaptive guessing
strategies.
\end{abstract}

\section{Introduction}
Mastermind is a two-player board game that was invented in 1970, and variants on its
basic four-spot, six-color structure have been studied extensively.
In the original game, there are two players: The codemaker and the codebreaker.
The codemaker initiates a game by constructing a 4-vector $p$ from the six available
colors, and the codebreaker attempts to guess the hidden vector in as few turns as
possible. A turn consists of two parts: First the codebreaker submits a query vector
$q$, which has the same form as a hidden vector. Second, the codemaker gives a
two-part response:
	\begin{enumerate}[label=\arabic*.]
	\item Black Hits: The number of correct colors in the correct spot
	(the number of positions $i$ such that $q_i=p_i$).
	\item White Hits: The number of correct colors in the incorrect spot, and which
	have not been used in another hit.
	\end{enumerate}
In 1976, Donald Knuth presented a greedy ``minimax'' algorithm, and he showed via
computer simulation that his algorithm always guesses the hidden vector in five turns
or fewer.\footnote{DK76} Moreover, the minimax algorithm for Mastermind is optimal in
the worst case---there is no algorithm that can guarantee to win in at most four turns.

There are multiple natural extensions to the original Mastermind game. Varying the
number of spots $n$ and colors $k$, as well as the relationship between $n$ and $k$,
is one such extension. In this paper, we also consider both allowing and prohibiting
repeated colors in vectors, and we examine both dual-color and black-only responses.
We explore another degree of freedom between adaptive strategies, in which the
codebreaker receives responses and adjusts the next guess accordingly, and non-adaptive
strategies, in which all guesses are submitted at the beginning of the
game and the hidden vector must be uniquely determined by the sequence of responses.
\footnote{Note that in a non-adaptive guessing strategy, our guesses need to
\textit{identify} the hidden vector rather than guess it outright. That is, a
non-adaptive strategy wins if the sequence of responses to the guess vectors allows
the codebreaker to distinguish between all possible hidden vectors. It is not
necessary that a winning set of non-adaptive guesses includes the hidden vector
itself.}

\subsection{Previous Work}
We build on the work of Grebinski and Kucherov; Chvatal; Doerr, Spohel, Thomas, and
Winzen; Erdos and Renyi; Knuth; Ko and Teng; Goodrich; and Bshouty. For a list of
previously established bounds on the query complexity of mastermind variants, please
see the appendix.

\subsection{Our Contribution}


\section{Adaptive Strategies}


\section{Non-Adaptive Strategies}


\clearpage
\section*{References}


\end{document}






