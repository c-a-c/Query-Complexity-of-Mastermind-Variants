\documentclass[12pt, a4paper]{article}
\author{Aaron Berger, Christopher Chute, Matthew Stone}
\title{Notes from Mastermind Research}
\usepackage[bottom=1.5in, left=1in, right=1in, top=1.1in]{geometry}
\usepackage{amsmath, amsthm, amssymb}
\newtheorem{lem}{Lemma}
\usepackage{graphicx}
\usepackage{setspace}
\usepackage{enumitem}
\usepackage{booktabs}
\usepackage{footnote}

% \begin{header footer formatting}
\usepackage{fancyhdr}
\setlength{\headheight}{48pt}
\pagestyle{fancyplain}
\lhead{Mastermind Project: \textit{Notes}\\\today{}}
\rhead{AB, CC, MS}
\rfoot{\thepage}
\cfoot{}
% \end{header footer formatting}�

% remove ``*'' for numbered theorems
\newtheorem{theorem}{Theorem}
\newtheorem{lemma}{Lemma}

% \begin{pretty i-hat and j-hat}
\newcommand*{\ihat}{\hat{\imath}}
\newcommand*{\jhat}{\hat{\jmath}}
\newcommand*{\jwidehat}{\widehat{\jmath}}
% \end{pretty i-hat and j-hat}

% \begin{pretty code or pseudocode}
\usepackage{listings}
\usepackage{courier}
\lstset{
         basicstyle=\footnotesize\ttfamily,
%        numbers=left,               
         numberstyle=\tiny,          
%        stepnumber=2,              
         numbersep=5pt,              
         tabsize=4,                 
         extendedchars=true,         
         breaklines=true,            
         keywordstyle=\color{red},
         frame=tblr,% Set the borders, tblr == (top, bottom, left, right)
%        keywordstyle=[1]\textbf,    
%        keywordstyle=[2]\textbf,   
%        keywordstyle=[3]\textbf,   
%        keywordstyle=[4]\textbf,  
         stringstyle=\color{white}\ttfamily,
         showspaces=false,          
         showtabs=false,           
         xleftmargin=17pt,
         framexleftmargin=17pt,
         framexrightmargin=5pt,
         framexbottommargin=4pt,
%        backgroundcolor=\color{red},
         showstringspaces=false            
 }
\lstloadlanguages{C}
% Example Usage:
% \lstinputlisting{pseudocode.txt}
% \end{pretty code or pseudocode}

% ********************************* END OF PREAMBLE ***********************************

\begin{document}
\section*{Mastermind with No Repeats, $n = k$: Lower Bound for Guessing Strategies}
	\begin{itemize}	
	\item Previous Bound: $n-n/log(n)$
	\item Our Bound:  There is no strategy that guarantees a win in $n-\log\log n$ turns.

	\end{itemize}
\section*{Lower Bound on Guessing Effectiveness for Permutation Game}
	\begin{itemize}
	\item Previous Work: None found
	\item Our work: In Mastermind with no repeats with $n$ spots and $k$ colors,
	for any set of possible remaining solutions, there exists a
	guess vector for which any response will eliminate at least $1/nk$ of the
	remaining solutions.
	This gives us the following asymptotic bounds:
		\begin{enumerate}[label=\roman*.]
		\item The minimax algorithm for Mastermind with $n$ spots, $k$
		colors, and no repeats takes at most $O(n^2k\log k)$ turns to find the
		hidden vector.
		
		\item In the same game where $n=k$, the minimax algorithm takes at most
		$O(n^3\log n)$ turns to find the hidden vector.
		\end{enumerate}
	
	\end{itemize}
	

\section*{Extension of Previous Section to Repeated Colors}
	
	\begin{itemize}
	\item Previous Work: None found
	\item Our Work: In a game of Mastermind with $n$ spots and $k$ colors, for any set of remaining
	solutions that can actually be achieved during a game of Mastermind,
	there is always a guess for which any response will eliminate at least
	$1/nk$ of the remaining solutions. \\
	This gives the same asymptotic bounds as above.
	
	\end{itemize}

\section*{Optimality of Mastermind for $(n, k) = (4, 6)$}
	\begin{itemize}
	\item Previous Work: [Doerr et al] cite Knuth as having proved this, but that appears to be wrong. As far as we know, this has not been proved yet.
	\item Our bound: It is not
	possible for a deterministic guessing strategy to guarantee finding the
	hidden vector in fewer than five turns.

	\end{itemize}
\clearpage
\section*{Lower Bound for Non-Adaptive Strategies Using Only Black Hits}

\begin{itemize}
	\item Previous Work: This is an adaptation of a bound for Mastermind with repeats by [Doerr et al] to Mastermind without repeats.
	\item Our Work: Every non-adaptive strategy for Mastermind with no repeats must submit at least
$O(n \log k)$ queries to uniquely identify all possible hidden vectors if given only black-hit responses.
\end{itemize}

\section*{Upper Bound on Non-Adaptive Strategies for Mastermind}
\begin{itemize}
	\item Previous Work: None found
	\item Our Work: There exists a set of at most $n k$ queries which uniquely determines every
possible hidden permutation (with or without repeats).
\end{itemize}

\section*{Non-Adaptive Lower Bound using Black and White Hits}
\begin{itemize}
	\item Previous Work: None found
	\item Our Work: For $k \ge n^2$, we get lower bound of $O(n\log k)$ turns for any non-adaptive strategy for
Mastermind without repeats using both black and white hits.
\end{itemize}
\end{document}






























